\problemname{Credit Card Payment}

%\setlength{\columnsep}{15pt}
\illustration{0.28}{card}{Picture from Wikimedia Commons}

Using credit cards for your purchases is convenient, but they have high
interest rates if you do not pay your balance in full each month.

The interest rate is commonly quoted in terms of ``annual percentage
rate'' (APR) which is then applied to the outstanding balance each
month.  The APR can be converted to a monthly interest rate $R$.  At the
end of each month, the monthly interest rate is applied to the
outstanding balance and the interest is added to the total balance.
Any payment made will be applied to the balance in the following
month.  The monthly interest is rounded to the nearest cent (rounding
up 0.5 cent and above) in the calculations.

You have unfortunately accumulated an outstanding balance $B$ at the
end of the month and you can only afford to pay up to some amount $M$
every month.  If you do not make any more purchases with the credit
card, what is the minimum number of payments needed to completely
eliminate the outstanding balance?  It is possible that you cannot pay
off the balance in 100 years (1200 payments).

\section*{Input}

The input consists of multiple test cases. The first line of input 
is a single integer, not more than $1000$, indicating the number
of test cases to follow.  
Each of the following lines specify the input for
one case.  Each line contains three positive real numbers separated by
single spaces: $R$, $B$, and $M$.  The real numbers have two digits after 
the decimal point, satisfying $R \leq 50.00$ and $B, M \leq 50000.00$.
$R$ is the monthly interest rate and is specified as a percentage.

\section*{Output}

For each case, display on a line the minimum number of payments needed
to eliminate the outstanding balance.  If this cannot be done in at
most 1200 payments, print instead \verb|impossible|.
