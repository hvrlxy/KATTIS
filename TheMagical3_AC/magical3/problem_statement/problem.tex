\problemname{The Magical 3}

\setlength{\columnsep}{15pt}
\illustration{0.3}{three.jpg}

% https://pixabay.com/en/three-3-number-design-collection-706895/
% CC0 public domain

There's no doubt about it, three is a magical number. Two's company,
but three's a crowd, no one ever talks about 2 blind mice, and there
are three members in an ACM ICPC team.

Even more magically, almost all integers can be represented as a
number that ends in 3 in some numeric base, sometimes in more than one
way.  Consider the number 11, which is represented as 13 in base 8 and
23 in base 4.  For this problem, you will find the smallest base for a
given number so that the number's representation in that base ends in
3.

\section*{Input}

Each line of the input contains one nonnegative integer $n$.  The
value $n = 0$ represents the end of the input and should not be
processed.  All input integers are less than $2^{31}$.  There are no more
than $1\,000$ nonzero values of $n$.

\section*{Output}

For each nonzero value of $n$ in the input, print on a single line the
smallest base for which the number has a representation that ends in 3.
If there is no such base, print instead ``\verb|No such base|''.
